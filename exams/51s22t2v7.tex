%  This exam is identical to 136s16t2makeup.

\documentclass[12pt]{article} 
\usepackage{amssymb,amsmath}
\usepackage{mdwlist}
%\usepackage{calculusmacros}
\usepackage{pstricks}
\usepackage{pst-grad,pst-plot,pst-coil}
\usepackage{graphicx}
\usepackage[margin=1in]{geometry}
%\pagestyle{empty}

%\oddsidemargin=0.0in
%\evensidemargin=0.0in
%\textwidth=6.5in
%\textheight=9.0in
%%\topmargin=-0.5in
%\thispagestyle{empty}
%
%\newtheorem{theorem}{Theorem}
%\newtheorem{lemma}[theorem]{Lemma}
%\newtheorem{corollary}[theorem]{Corollary}
%\newtheorem{proposition}[theorem]{Proposition}
%
%\newtheorem{definition}[theorem]{Definition}
%\newtheorem{example}[theorem]{Example}
%
%
%\newcommand{\xo}{x_0}


%\newcommand{\vw}{V\times W}
%\newcommand{\cA}{\mathcal{A}}
%\newcommand{\cC}{\mathcal{C}}
%\newcommand{\cL}{\mathcal{L}}
%
%\newcommand{\cl}{\operatorname{cl}}
%\newcommand{\intt}{\operatorname{int}}
%
%\newcommand{\norm}[1]{\|#1\|}
%
%\newcommand{\bel}[1]{\begin{equation}\label{#1}}
%\newcommand{\be}{\begin{equation}}
%\newcommand{\ee}{\end{equation}}
%
%\newcommand{\deriv}[2]{\frac{d #1}{d #2}}
%\newcommand{\pderiv}[2]{\frac{\partial #1}{\partial #2}}
%\newcommand{\pd}[1]{\partial /\partial{#1}}
%\newcommand{\pdiffx}[1]{\frac{\partial\ }{\partial x_{#1}}}
%\newcommand{\pdiffy}[1]{\frac{\partial\ }{\partial y_{#1}}}
%\newcommand{\pdiffz}[1]{\frac{\partial\ }{\partial z_{#1}}}
%\newcommand{\pdiffzbar}[1]{\frac{\partial\ }{\partial \bar{z}_{#1}}}
%
%\newcommand{\cinf}{C^{\infty}}
%
%%\newcommand{\qed}{\vrule height 2ex depth 0pt width 0.5em}
%
%\newcommand{\ra}{\rangle}
%\newcommand{\supp}{\operatorname{supp}}


\newcommand\ms {\medskip}
\newcommand\bs {\bigskip}
\newcommand\dsp{\displaystyle}
\newcommand\cc{{\mathcal{C}}}
\newcommand\rr{{\mathbb{R}}}
\newcommand\rone{{\mathbb{R}}}
\newcommand\rtwo{{\mathbb{ R}^2}}
\newcommand\rn{{\mathbb{R}^n}}
\newcommand\rrm{{\mathbb{R}^m}}
\newcommand\V{\mathcal{V}}

\newcommand\nn{{\mathbb{N}}}
\renewcommand\ss{{\mathbb{S}}}

%\newcommand{\cl}{\operatorname{cl}}
\newcommand{\bd}{\operatorname{bd}}
\newcommand\zz{{\mathbb{Z}}}
\newcommand\ii{{\mathbb{I}}}
\newcommand\qq{{\mathbb{Q}}}
\newcommand\eps{{\epsilon}}
\newcommand\rar{\rightarrow}
\newcommand\st{\hspace{0.4mm}\big|\hspace{0.4mm}}
\newcommand\x{\mathbf{x}}
\newcommand\y{\mathbf{y}}
\newcommand\zero{\mathbf{0}}
\newcommand\xoyo{(x_0,y_0)}

\def\dsp{\displaystyle}
\def\rr{{\mathbb R}}
\def\R{{\mathbb R}}
\def\rone{{\mathbb R}}
\def\rtwo{{\mathbb R^2}}
\def\rn{{\mathbb R^n}}
\def\rm{{\mathbb R^m}}
\def\nn{{\mathbb N}}
\def\zz{{\mathbb Z}}
\def\Q{{\mathbb Q}}
\def\rar{\rightarrow}
\def\st{\,\big|\,}
\def\eps{\epsilon}
\def\acc{\operatorname{acc}}
\def\sup{\operatorname{sup}}
\def\inf{\operatorname{inf}}
\def\intt{\operatorname{int}}
\def\bd{\operatorname{bd}}
\def\cl{\operatorname{cl}}
\def\intab{\int_a^b}
\def\gap{\operatorname{gap}}
\def\vh{\vec {\mathbf{h}}}
\def\vp{\vec {\mathbf{p}}}
\def\vE{\vec {\mathbf{E}}}
\def\vx{\vec {\mathbf{x}}}
\def\vv{\vec {\mathbf{v}}}
\def\vy{\vec {\mathbf{y}}}
\def\vL{\vec {\mathbf{L}}}
\def\v0{\vec {\mathbf{0}}}

\def\vM{\vec {\mathbf{M}}}
\def\ms{\medskip}
\def\ss{\smallskip}
\def\bfu{\mathbf u}
\def\bfv{\mathbf v}


\newcommand{\bE}{\vec{E}}
\newcommand{\bF}{{\bf F}}
\newcommand{\bu}{\vec{\mathbf{u}}}
\newcommand{\bv}{\vec{\mathbf{v}}}
\newcommand{\bw}{{\bf w}}
\newcommand{\ba}{{\bf a}}
\newcommand{\bb}{{\bf b}}
\newcommand{\bh}{{\bf h}}
\newcommand{\bx}{\vec{x}}
\newcommand{\bo}{{\bf 0}}
\newcommand{\bp}{{\bf p}}
\newcommand{\br}{{\bf r}}
\newcommand{\bn}{{\bf n}}
\newcommand{\bT}{{\bf T}}
\newcommand{\bk}{{\bf k}}


\setlength\parindent{25pt}

\usepackage{color}
\newcommand{\tred}[1]{{\color{black}{#1}}}

\parindent=0pt
\parskip=\baselineskip

\begin{document}


\noindent
Carefully PRINT your full name:  \framebox(250,30){}


\begin{center}
Math 51 \hfill Differential Equations \hfill April 11, 2022\\
\phantom{noon--1:20 p.m.} \hfill Exam 2 (100 points) \hfill noon--1:20 p.m.
\end{center}

\medskip

There are seven problems on the exam.

You may not use calculators, books or notes during the exam. All electronic devices (including
your phones) must be silenced and put away for the duration of the
exam.

After finishing your exam, you will submit this exam booklet. We will scan your submission and
upload it to Gradescope for marking (you do not need to take images of your exam). You should
write your name at the top of each page, as indicated (especially if you remove the staples from
your exam booklet).

For the partial credit problems, always show your work. Try to fit this work in the available space
if possible. There is a blank page at the back of your exam for use as scratch paper. If you need
more space for a solution, please write clearly in the indicated space that your solution continues
later.

* * * * * * * * * * * * * * * * * * * * * * * * * * * * * * * * * * *
* * * * *

Please sign the pledge below. With your signature, you pledge that
you have neither given nor received assistance on this exam.


\bigskip
\bigskip
Signature:  \framebox(250,30){}

\newpage

% 1.
\begin{enumerate}
   
%\setcounter{enumi}{1}

\item (14 points) True-False and Multiple Choice.
  \begin{enumerate}
    \setlength{\itemsep}{20mm}
    
    \item (2 pts.) True or False (circle your choice).  Any set of vectors that includes the zero
      vector $\mathbf{0}$ is linearly dependent.
      
     

\item (2 pts.) True or False (circle your choice).  If $\vp_1$ and $\vp_2$ are solutions of the nonhomogeneous
  system $D\vx = A\vx + \vE(t)$, then $\vp_1 - \vp_2$ is a solution of
  the related homogeneous system $D\vx = A\vx$.
  
 

  \item (2 pts.) True or False (circle your choice).  Assume that all the functions in this question are differentiable.  Let $\vx_1, \ldots, \vx_n$ be solutions of the \emph{nonhomogeneous}
  linear system $D\vx = A\vx + \vE(t)$ on an interval $I$ and let
  $t_0$ be a point in $I$.
  Then $\vx_1, \ldots, \vx_n$ generate the general solution of the given
  system if and only if the Wronskian $W[\vx_1, \ldots, \vx_n](t_0)
  \ne 0$.
  


  \item (2 pts.) True or False (circle your choice). Two vectors $\bv_1$ and $\bv_2$ are linearly dependent if and
  only if one of them is a constant multiple of the other. 


\item (2 pts.) True or False (circle your choice).  Three vectors $\bv_1$, $\bv_2$ and
  $\bv_3$ are linearly dependent if and only if some vector is a constant multiple of another vector.


\item  (2 pts.)  Which of the following formulas for $\vx$ gives the
  general solution to the linear homogeneous ode $(D^2+1)^3\vx\ =\v0$?
  Circle your choice.
    \begin{enumerate}
\item[A.] $c_1 \cos t +c_2 \sin t$
\item[B.] $c_1t^2 \cos t +c_2t^2 \sin t$
\item[C.] $c_1t^2 \cos t +c_2t^2 \sin t +c_3t \cos t +c_4t \sin t +c_5
  \cos t +c_6 \sin t$
\item[D.] None of the above.
\end{enumerate}


\newpage
\vspace*{-.5in}
\rightline{Your full name:  \framebox(250,30){}}
\item (2 pts.) For which of the following expressions for $E(t)$ does the method of
undetermined coefficients \textbf{not} apply when solving the linear nonhomogeneous
ode $Lx = E(t)$?  Circle your choice.
\[
3t^4, \qquad \sin t, \qquad 2t^3e^{-4t} \cos 5t, \qquad \ln t.
\]

\begin{enumerate}
  \item[A.] Only $3t^4$.
\item[B.] Only $\sin t$.
\item[C.] Only $2t^3e^{-4t} \cos 5t$.
\item[D.] Only $\ln t$.
\item[E.] The method of undetermined coefficients does not apply for at least two of the four functions.
  \item[F.] The method of undetermined coefficients applies for all four functions.
  \end{enumerate}
  
\end{enumerate}

\vspace*{.5in}

\item (16 points) Short-Answer Questions.
  \begin{enumerate}

      \item (3 pts.) The matrix 
\begin{equation*}
          A = \left[
            \begin{array}{rrr}
              1 & 0 & 0 \\
              1 & 1 & 1\\
              1 & 0 & 1
            \end{array}
          \right]
        \end{equation*}
has eigenvalue 1 with multiplicity 3.  Find three linearly independent
generalized eigenvectors.  No explanation is required.  (\textit{Hint}:  Approached correctly, this
problem does not need any computation.)

\vfill

\newpage

\item (5 pts.) Write down an annihilator of smallest possible order
  with real coefficients for the function $3e^t + 2te^{-t} + \sin t$.
  No work or explanation is required.
  
  \vfill
 

\item (3 pts.) Suppose
  \begin{equation*}
          \begin{array}{rrrrrrl}
              2 u_1&+& 5 u_2&+& 3 u_3&= & a\\
               u_1& -&2 u_2& +& u_3& = & b\\
               u_1& +&4 u_2& - & u_3 & = & c,
          \end{array}
          \end{equation*}
          where it is given that the determinant of the coefficient matrix is nonzero.
          Write down the formula for $u_3$ in terms of determinants.
          Do not evaluate the determinants.

\vfill
    

 
%\setcounter{page}{1}

\item (5 points) Given that $\vh_1(t) = \left[
            \begin{array}{r}
              e^{-t}  \\
              -e^{-t} 
            \end{array}
          \right]$  and $\vh_2(t) =\left[
            \begin{array}{r}
              e^{-2t}  \\
              -2e^{-2t} 
            \end{array}
          \right]$ are solutions of $D\vx = A\vx$, where $A = \left[
            \begin{array}{rr}
              0 & 1  \\
              -2 & -3 
            \end{array}
          \right]$, determine whether or not the general solution is
          $\vx(t) = c_1\vh_1(t) + c_2 \vh_2(t)$.
          Show your work.
       \end{enumerate} 
        
          \vfill
          
%           \newpage

% \item (10 points) Find the eigenvalues of the matrix
%   \begin{equation*}
%           A = \left[
%             \begin{array}{rrr}
%               1 & 2 & 3 \\
%               0 & 2 & 0 \\
%               0 & 1 & 2
%             \end{array}
%           \right]
%         \end{equation*}
% and for each eigenvalue, find as many linearly independent
%         eigenvectors as possible.
        
        \newpage
\vspace*{-.5in}
\rightline{Your full name:  \framebox(250,30){}}


\item (12 points)
        \begin{enumerate}
          \item[(a)] Convert the differential equation
        \[
         x'''-e^t x''-4tx + x = e^{2t}
        \]
        into a linear system of three equations in three unknowns
        $x_1, x_2, x_3$.
        \item[(b)] Write the linear system in the form $D\vx = A(t)
          \vx + \vE(t)$ for some matrix $A(t)$ and vector $\vE(t)$.
        \end{enumerate}
        
        \newpage

%Sec. 3.8, \#2
\item (10 points) Suppose $3+2i, 3-2i$ are eigenvalues of the $2 \times 2$
            matrix $A$ with corresponding eigenvectors \tred{
            $\begin{bmatrix} i\\1\end{bmatrix},
            \begin{bmatrix} -i\\ \phantom{-}1\end{bmatrix}$,}
            respectively.
            Write down two linearly independent real
            solutions of $D\vx = A \vx$.  Show your work and simplify
            your answers.
            
            \newpage
 \vspace*{-.5in}
\rightline{Your full name:  \framebox(250,30){}}
        

\item (15 points) Find the general solution of $4x''-4x'+x = \frac{8}{t^2} e^{t/2}$
  for $t > 0$, given that two solutions to the associated homogeneous
  equation are $\vh_1 = e^{t/2}$, $\vh_2 = t e^{t/2}$.

          \newpage
          % S3.9, p. 296, #3
         \item (13 points) The matrix $A$ below has an eigenvalue $\lambda$ and a generalized eigenvector $\vv$ as follows:
          \begin{equation*}
          A = \left[
            \begin{array}{rr}
              1 & 2  \\     0 & 1 
            \end{array}
          \right], \qquad \lambda =1, \qquad \vv = \begin{bmatrix} 1 \\ 0 \end{bmatrix}.
        \end{equation*}
Find the general solution of $D\vx = A \vx$.

\newpage

\vspace*{-.5in}
\rightline{Your full name:  \framebox(250,30){}}


% S2.7, p. 146, #14
        \item (10 points) Make a \textit{simplified} guess for a particular solution of
    the differential equation \tred{
    \[
     (D+2)^3 (D^2+1)^2 \,\vx = t e^{-2t} + \sin t.
    \] }
    Do not solve for the coefficients.
  \end{enumerate}
  
  \end{document}
      
