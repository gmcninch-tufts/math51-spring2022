\documentclass[12pt]{article} 
\usepackage{amssymb,amsmath}
\usepackage[margin=.9in]{geometry}

\setlength\parindent{25pt}

%\usepackage{color}
%\newcommand{\tred}[1]{{\color{red}{#1}}}


\begin{document}


\noindent
Practicum Session: \framebox(30,30){} \quad PRINT your name:  \framebox(230,30){}\\

\smallskip
\noindent
Section Instructor: \framebox(100,30){} (Choose from Hasselblatt, McNinch, Smith, Tu)


\begin{center}
Math 51 \hfill Differential Equations \hfill February 14, 2022\\
\hfill ~~~~~~~~~Exam 1 (100 points) \hfill noon--1:20 p.m.
\end{center}

\medskip

This is a closed-book exam.  No books, notes, or calculators are permitted.
% At the
% end of the exam, scan your exam as a single pdf file
% and upload them to Gradescope.

At the end of the exam you are required to sign a pledge that you have not
cheated.  
Students found violating the pledge will be reported to the Dean of Student Affairs.

You must show your work in all the questions that require
calculations, explanations, 
or proofs.  Simplify all answers.



%\newpage

\begin{enumerate}
 \setlength{\itemsep}{5mm}

 % 1.  

\item (21 points) Short-answer questions.  Two points per question
  except in (a).  No work needs to be shown, as only the answer will be graded.

  \begin{enumerate}
    \setlength{\itemsep}{5mm}


    \item (3 points) Consider the differential equation
      \[
        (t-3)^2 x'' + \frac{1}{t-2} x' + 2x = 0.
        \]
        Find the largest open interval containing $t=0$ on which the
        o.d.e.\ is normal.
        
        \hfill \framebox[180px]{3*\strut}

         \item The order of the o.d.e. $t^3 (x'')^3 + 3x' +tx^4 = 0$ is

      \hfill \framebox(60,30){}


        \item  True or False.  Let $h_1, h_2, h_3$ be solutions of a normal
          third-order linear differential equation on an open
      interval $I$.  Then the Wronskian $W[h_1, h_2, h_3](t_0) = 0$
      at one point $t_0 \in I$ if and only if $W[h_1, h_2, h_3](t)
      =0$ at every point $t \in I$.

      \hfill \framebox(60,30){}

      \item True or False.  Let $h_1, h_2, h_3$ be arbitrary infinitely differentiable
        functions on an open interval $I$.  Then the Wronskian $W[h_1, h_2, h_3](t_0) = 0$
      at one point $t_0 \in I$ if and only if $h_1, h_2, h_3$
      are linearly dependent on $I$.

      \hfill \framebox(60,30){}

           \item True or False.  The linear o.d.e $t^3 x'' + 3x' + x + t
          = 0$ is homogeneous.

          \hfill \framebox(60,30){}

          


            \item True or False.  If $h_1(t), h_2(t), h_3(t)$ are
              three solutions of the differential equation $x'' + x +
              1 = 0$ on an interval $I$, then they cannot be
              linearly independent on $I$.

              \hfill \framebox(60,30){}

            \item True or False.  Let $P(r), Q(r)$ be two polynomials
              and $F(r) = P(r) Q(r)$.   If $x= h(t)$ is a solution of
              the differential equation $P(D)x=0$, then it is necessarily a
              solution of the differential equation $F(D) x = 0$.

              \hfill \framebox(60,30){}

              \item Yes or No.  Suppose $h_1(t)$ and $h_2(t)$ are any two
            solutions of the differential equation $t^3 x'' + tx' +
            (t^2-1)x = 1$.  Is $h_1(t) + h_2(t)$ necessarily a solution of
            this differential equation?

            \hfill \framebox(60,30){}

            \item Yes or No.  Suppose $h_1(t)$ and $h_2(t)$ are two any
            solutions of the differential equation $t^3 x'' + x' +
            x^2 = 0$.  Is $h_1(t) + h_2(t)$  necessarily a solution of
            this differential equation?

            \hfill \framebox(60,30){}


              \item Multiple-Choice.  Let $L(x) = x''' - 3x' +x =0$.  Suppose $h_1(t),
                h_2(t), h_3(t)$ are solutions of $L(x)=0$ on $(-\infty,
                \infty)$ and the
                Wronskian $W[h_1, h_2, h_3](1) \ne 0$.
                Which of the following statements is true?
                \smallskip
                \begin{enumerate}
                  \item[I.] $h_1(t), h_2(t), h_3(t)$ generate the
                    general solution of $L(x)=0$ on $(-\infty,\infty)$.
                    \item[II.] $h_2(t), h_2(t), h_3(t)$ are linearly
                      independent on $(-\infty,\infty)$.
                    \end{enumerate}

                    \medskip
                    Write one of A, B, C, D, or E in the box.

                    \medskip

                    \begin{enumerate}
                  \item[A.]  Only I is true.
                  \item[B.] Only II is true.
                  \item[C.] Both I and II are true.
                  \item[D.] Neither I nor II is true.
                    \item[E.] It is not possible to determine the
                      truth or falsity of I or II from the given information.
                  \end{enumerate}
                  

            \hfill \framebox(60,30){}

            
\end{enumerate}
    
\newpage
  

  \item (12 points) Determine whether each collection of functions is linearly
    independent or dependent on the interval $-\infty < t < \infty$.
    Justify your answer.
    \begin{enumerate}
    \item $f_1(t) = e^t, \quad f_2(t) = t e^t, \quad f_3(t) = 1$.
    \item $g_1(t) = t^2, \quad g_2(t) = - t^2$.
      \end{enumerate}

      %\newpage

\item (20 points)  Write the general solution for each differential
  equation below.
  \begin{enumerate}
  \item $D(D^2 -9) (D^2 -2D-1) x = 0$.
  \item $(D^2 -2D -15)^2 x = 0$.
  \end{enumerate}

  %\newpage

\item (6 points)
\begin{enumerate}
\item[(a)]  (3 pts)  Calculate
  \[
    \det \begin{bmatrix}
      3 & 0 &2 \\
      5 & 1 & 0 \\
      2 & -1 & 1
    \end{bmatrix}.
    \]

      \item[(b)] (3 pts) Suppose
          \begin{align*}
            u_1 + 2 u_2 + 3 u_3 &= a,\\
            3u_1 + 2 u_2 + 1 u_3 &= b,\\
            5 u_1 - 2 u_2 + 2 u_3 &= c.
          \end{align*}
          where it is given that the determinant of the coefficient matrix is nonzero.
          Write down the formula for $u_3$ in terms of determinants.
          Do not evaluate the determinants.

          % \item[(c)] (5 pts) Find a rational root of the polynomial
          %   $P(D) = 2 D^3 +3 D^2 - 1$.

        \end{enumerate}
        
%\newpage

          \item (16 points) Use variation of parameters to find the general solution of
  \[
     \frac{dx}{dt} + \frac{2}{t} x = \frac{e^{2t}}{t^2}, \quad t >0.
  \]
  (If you use a different method, you will get at most 10 points.)

  %\newpage

  \item (10 points) Using the exponential shift formula $P(D)[e^{\lambda t}y] = e^{\lambda t} [P(D+\lambda) y]$, evaluate $(D+5)^3 \big[(t^3 - 4t^2 + 2t -4) e^{-5t}\big]$.

    %\newpage

  \item (15 points) Consider the equation $t^2 x'' - t x' + x =0$ on
    the interval $(0, \infty)$.
    \begin{enumerate}
      \item Show that $h_1(t) = t$ and $h_2(t) = t \ln t$ are
        solutions on this interval.
        \item Do these two solutions generate the
          general solution?  Explain your reasoning.
        \end{enumerate}

      \end{enumerate}

      \vfill

\noindent
PLEDGE:  I pledge that during this exam I have neither given nor received assistance or cheated in any other way.

\bigskip
\noindent
Signature: \framebox(250,30){}

\begin{center}
(End of Exam)
\end{center}

      \end{document}
 


% Find a minimal annihilator for the function
%     $5 e^{3t} + 4 e^{-2t} + 3 e^{-5}\cos 3t$ or state why there isn't
%     one.


  
 
  
