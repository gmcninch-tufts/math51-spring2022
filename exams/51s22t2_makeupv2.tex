%  This exam is identical to 136s16t2makeup.

\documentclass[12pt]{article} 
\usepackage{amssymb,amsmath}
\usepackage{mdwlist}
%\usepackage{calculusmacros}
\usepackage{pstricks}
\usepackage{pst-grad,pst-plot,pst-coil}
\usepackage{graphicx}
\usepackage[margin=1.1in]{geometry}
%\pagestyle{empty}

%\oddsidemargin=0.0in
%\evensidemargin=0.0in
%\textwidth=6.5in
%\textheight=9.0in
%%\topmargin=-0.5in
%\thispagestyle{empty}
%
%\newtheorem{theorem}{Theorem}
%\newtheorem{lemma}[theorem]{Lemma}
%\newtheorem{corollary}[theorem]{Corollary}
%\newtheorem{proposition}[theorem]{Proposition}
%
%\newtheorem{definition}[theorem]{Definition}
%\newtheorem{example}[theorem]{Example}
%
%
%\newcommand{\xo}{x_0}


%\newcommand{\vw}{V\times W}
%\newcommand{\cA}{\mathcal{A}}
%\newcommand{\cC}{\mathcal{C}}
%\newcommand{\cL}{\mathcal{L}}
%
%\newcommand{\cl}{\operatorname{cl}}
%\newcommand{\intt}{\operatorname{int}}
%
%\newcommand{\norm}[1]{\|#1\|}
%
%\newcommand{\bel}[1]{\begin{equation}\label{#1}}
%\newcommand{\be}{\begin{equation}}
%\newcommand{\ee}{\end{equation}}
%
%\newcommand{\deriv}[2]{\frac{d #1}{d #2}}
%\newcommand{\pderiv}[2]{\frac{\partial #1}{\partial #2}}
%\newcommand{\pd}[1]{\partial /\partial{#1}}
%\newcommand{\pdiffx}[1]{\frac{\partial\ }{\partial x_{#1}}}
%\newcommand{\pdiffy}[1]{\frac{\partial\ }{\partial y_{#1}}}
%\newcommand{\pdiffz}[1]{\frac{\partial\ }{\partial z_{#1}}}
%\newcommand{\pdiffzbar}[1]{\frac{\partial\ }{\partial \bar{z}_{#1}}}
%
%\newcommand{\cinf}{C^{\infty}}
%
%%\newcommand{\qed}{\vrule height 2ex depth 0pt width 0.5em}
%
%\newcommand{\ra}{\rangle}
%\newcommand{\supp}{\operatorname{supp}}


\newcommand\ms {\medskip}
\newcommand\bs {\bigskip}
\newcommand\dsp{\displaystyle}
\newcommand\cc{{\mathcal{C}}}
\newcommand\rr{{\mathbb{R}}}
\newcommand\rone{{\mathbb{R}}}
\newcommand\rtwo{{\mathbb{ R}^2}}
\newcommand\rn{{\mathbb{R}^n}}
\newcommand\rrm{{\mathbb{R}^m}}
\newcommand\V{\mathcal{V}}

\newcommand\nn{{\mathbb{N}}}
\renewcommand\ss{{\mathbb{S}}}

%\newcommand{\cl}{\operatorname{cl}}
\newcommand{\bd}{\operatorname{bd}}
\newcommand\zz{{\mathbb{Z}}}
\newcommand\ii{{\mathbb{I}}}
\newcommand\qq{{\mathbb{Q}}}
\newcommand\eps{{\epsilon}}
\newcommand\rar{\rightarrow}
\newcommand\st{\hspace{0.4mm}\big|\hspace{0.4mm}}
\newcommand\x{\mathbf{x}}
\newcommand\y{\mathbf{y}}
\newcommand\zero{\mathbf{0}}
\newcommand\xoyo{(x_0,y_0)}

\def\dsp{\displaystyle}
\def\rr{{\mathbb R}}
\def\R{{\mathbb R}}
\def\rone{{\mathbb R}}
\def\rtwo{{\mathbb R^2}}
\def\rn{{\mathbb R^n}}
\def\rm{{\mathbb R^m}}
\def\nn{{\mathbb N}}
\def\zz{{\mathbb Z}}
\def\Q{{\mathbb Q}}
\def\rar{\rightarrow}
\def\st{\,\big|\,}
\def\eps{\epsilon}
\def\acc{\operatorname{acc}}
\def\sup{\operatorname{sup}}
\def\inf{\operatorname{inf}}
\def\intt{\operatorname{int}}
\def\bd{\operatorname{bd}}
\def\cl{\operatorname{cl}}
\def\intab{\int_a^b}
\def\gap{\operatorname{gap}}
\def\vx{\vec {\mathbf{x}}}
\def\vy{\vec {\mathbf{y}}}
\def\vL{\vec {\mathbf{L}}}
\def\vM{\vec {\mathbf{M}}}
\def\ms{\medskip}
\def\ss{\smallskip}
\def\bfu{\mathbf u}
\def\bfv{\mathbf v}


\newcommand{\bE}{\vec{E}}
\newcommand{\bF}{{\bf F}}
\newcommand{\bu}{\vec{u}}
\newcommand{\bv}{\vec{v}}
\newcommand{\bw}{{\bf w}}
\newcommand{\ba}{{\bf a}}
\newcommand{\bb}{{\bf b}}
\newcommand{\bh}{{\bf h}}
\newcommand{\bx}{\vec{x}}
\newcommand{\bo}{{\bf 0}}
\newcommand{\bp}{{\bf p}}
\newcommand{\br}{{\bf r}}
\newcommand{\bn}{{\bf n}}
\newcommand{\bT}{{\bf T}}
\newcommand{\bk}{{\bf k}}


\newcommand{\Real}{\operatorname{Re}}
\newcommand{\Imag}{\operatorname{Im}}

\setlength\parindent{25pt}

\usepackage{color}
\newcommand{\tred}[1]{{\color{red}{#1}}}


\begin{document}


\noindent
Carefully PRINT your full name:  \framebox(250,30){}


\begin{center}
Math 51 \hfill Differential Equations \hfill Spring 2022\\
Alternate Exam 2 (90 pts.+ 10 bonus pts) 
\end{center}

\medskip


You may not use calculators, books or notes during the exam. All electronic devices (including
your phones) must be silenced and put away for the duration of the
exam.

After finishing your exam, you will submit this exam booklet. We will scan your submission and
upload it to Gradescope for marking (you do not need to take images of your exam). You should
write your name at the top of each page, as indicated (especially if you remove the staples from
your exam booklet).

For the partial credit problems, always show your work. Try to fit this work in the available space
if possible.
% There is a blank page at the back of your exam for use as scratch paper. If you need
% more space for a solution, please write clearly in the indicated space that your solution continues
% later.

* * * * * * * * * * * * * * * * * * * * * * * * * * * * * * * * * * *
* * * * *

Please sign the pledge below. With your signature, you pledge that
you have neither given nor received assistance on this exam.


\bigskip
\bigskip
Signature:  \framebox(250,30){}

\newpage


% 1.
\begin{enumerate}

\item (16 points)  True-false and multiple choice.  Circle the correct choice.
  \begin{enumerate}
  \setlength{\itemsep}{10mm}
  

\item Let $\bh_1, \ldots, \bh_n$ be $n$ solutions of an order-$n$
  linear system $D\bx=A\bx$ on an
  interval $I$.  Is it possible that the Wronkisan $W[\bh_1, \ldots,
  \bh_n](t)$ is 0 at one point $t_0$ of $I$ but nonzero at another
  point $t_1$ of $I$?
  
  \bigskip
  
      \begin{enumerate}
      \item[A.] Yes
      \item[B.] No
      \end{enumerate}


     \item True or False.  An $n\times n$ real matrix must have $n$ linearly
       independent eigenvectors, some of which may be complex.


       \item True or False.  Let $A$ be a matrix of real numbers.  The linear system
         $D\bx = A \bx$ must have $n$ linearly independent real
         solutions.

         \item True or False. Let $A$ be a matrix of real numbers. If $\bx$ is a
           complex solution of $D\bx = A\bx$, then both $\Real \bx$ and
           $\Imag \bx$ are real solutions of $D\bx = A\bx$.

          \item True or False.  Five vectors in $\R^5$ are linearly independent if
            and only if they generate (span) $\R^5$.

            \item True or False.   Let $A$ be an $n \times n$ matrix with an eigenvalue
              $\lambda$ of multiplicity $3$.  An eigenvector
              corresponding to $\lambda$ is also a generalized eigenvector.

          \item True or False.  Let $A$ be an $n \times n$ real matrix.  The general
            solution of $D\bx = A\bx$ can be generated by fewer than
            $n$ solutions.

          \item True or False. For every eigenvalue $\lambda$ of an $n \times n$
            matrix, there must be a corresponding eigenvector.
\end{enumerate}
            \newpage
            
\item (8 pts) Write down an annihilator of smallest
      possible order with real coefficients for the function
      \[
        4 t e^{3t}+ t^2 e^{2t} \sin t.
      \]


\newpage
      
\item  (10 points) Make a \textit{simplified} guess for a particular solution of
    the differential equation
    \[
      (D+2)^7 (D^2+1)^6 x = t e^{-2t} + \cos t.
    \]
    Do not solve for the coefficients.

%\setcounter{page}{1}


\newpage

% \setlength{\itemsep}{5mm}
%\setcounter{enumi}{1}

      \item (14 points)
        \begin{enumerate}
          \item[(a)] Convert the differential equation
        \[
          x''' - t^2 x'' + \pi x' + t x = \sin t
        \]
        into a linear system of three equations in three unknowns
        $x_1, x_2, x_3$.
        \item[(b)] Write the linear system in the form $D\bx = A(t)
          \bx + \bE(t)$ for some matrix $A(t)$ and vector $\bE(t)$.
        \end{enumerate}



        \newpage

%      \item (14 points)  Find the eigenvalues of the matrix
%        \begin{equation*}
%          A = \left[
%            \begin{array}{rrr}
%              1 & -1 & -1 \\
%              0 & 0 & -1 \\
%              0 & 0 & 1
%            \end{array}
%          \right]
%        \end{equation*}
%        and for each eigenvalue, find as many linearly indepedent
%        eigenvectors as possible.
%
%        \newpage

      \item (15 points) 
      \begin{enumerate}
      \item[(a)] (10 pts) The matrix $A=\begin{bmatrix} 1 & -1 \\ 1 &\phantom{-}3
          \end{bmatrix}$ has eigenvalue $\lambda =2$ of multiplicity
          2.
          It has an eigenvector
          $\bv =\begin{bmatrix} \phantom{-}1 \\ -1\end{bmatrix}$
          and a generalized eigenvector
          $\bu=\begin{bmatrix} \phantom{-}0 \\ -1\end{bmatrix}$.
          Using these two vectors, write down two solutions of $D\bx = A\bx$ that generate the
          general solution.  
          
     \vspace{2in}
     
     \item[(b)] (5 pts) Suppose we have found three solutions of a linear system $D\bx = A\bx$:
     \begin{gather*}
     e^t \left( \begin{bmatrix} 1 \\0 \\0 \end{bmatrix} +      t \begin{bmatrix} -1 \\ \phantom{-}0 \\ \phantom{-} 1   \end{bmatrix}
      + \frac{1}{2} t^2 \begin{bmatrix} 1\\1\\1\end{bmatrix}\right), \quad
 e^t \left( \begin{bmatrix} 0 \\1 \\0 \end{bmatrix} +
      t \begin{bmatrix} \phantom{-}1 \\ -1 \\ -3   \end{bmatrix}
      - t^2 \begin{bmatrix} 1\\1\\1\end{bmatrix}\right),  \\  
      e^t \left( \begin{bmatrix} 0 \\0 \\1 \end{bmatrix} +
      t \begin{bmatrix} 0 \\ 1 \\ 2   \end{bmatrix}
      + \frac{1}{2} t^2 \begin{bmatrix} 1\\1\\1\end{bmatrix}\right).
  \end{gather*}
  Explain why these three solutions generate the general solution of
  $D\bx = A\bx$.
      
      
     \end{enumerate}     
          
          
          
          
          
          \newpage


          \item (12 points) Suppose $i, -i$ are eigenvalues of the $2 \times 2$
            matrix $A$ with corresponding eigenvectors
            $\begin{bmatrix} 1\\i\end{bmatrix},
            \begin{bmatrix} \phantom{-}1\\-i\end{bmatrix}$,
            respectively.
            Write down two linearly independent real
            solutions of $D\vx = A \vx$.  Show your work and simplify
            your answers.

            \newpage

    \item (15 points) Find the general solution of the differential
      equation
      \[
        5 x'' -10 x' + 5x = t^{1/5} e^t,
      \]
      given that two independent solutions of the related homogeneous
      equation are $e^t$ and $te^t$.

     \newpage
      (Continuation of Question 7)

     % \newpage

    % \item (10 points)
    %   Decide if the vectors
    %   \[
    %      \bu_1 = \begin{bmatrix} 1\\1\\1\\2 \end{bmatrix}, \quad
    %       \bu_2 = \begin{bmatrix} 2\\0\\3\\1\end{bmatrix}, \quad
    %     \bu_3 = \begin{bmatrix} 0\\2\\1\\3\end{bmatrix}
    %     \]
    %   are linearly independent.  You must show your work.
           
   
      
        
    % \item (20 points)  Given that the characteristic polynomial of the
    %   matrix
    %   \[
    %     A = \begin{bmatrix}
    %       1 & 2 & 3\\
    %       0 & 2 & 0 \\
    %       0 & 1 & 2
    %     \end{bmatrix}
    %   \]
    %   is $\det (A - \lambda I) = (1-\lambda)(2- \lambda)^2$, find the
    %   general solution of the system $D\bx = A \bx$.

   
      

    % \item (15 points)  Given that
    %   \[
    %     A = \begin{bmatrix}
    %       1 & -1\\
    %       5 & -3
    %     \end{bmatrix},
    %   \]
    %   find the general solution of the system $D\bx = A\bx$.
    %   (\textit{Hint}: The characteristic polynomial has complex roots.)

    %\newpage
    
  
   
  %\item (15 points)

    % Given that two linearly independent solutions of $D\bx = A\bx$
    %   are
    %   \[
    %      \begin{bmatrix} 2 e^t \\ e^t\end{bmatrix}, \quad
    %     \begin{bmatrix} e^{2t} \\
    %       e^{2t}\end{bmatrix},
    %   \]
    %   find the general solution of $D\bx = A\bx + \begin{bmatrix}
    %    e^t\\ e^t\end{bmatrix}$.
      
 %    \newpage
% (Continuation of Question 8)

 
   \end{enumerate}

\vfill

\begin{center}
(End of Exam)
\end{center}

\end{document}
