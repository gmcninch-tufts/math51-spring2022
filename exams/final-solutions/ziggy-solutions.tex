\documentclass[11pt]{amsart}
\usepackage{geometry}                % See geometry.pdf to learn the layout options. There are lots.
\geometry{letterpaper}                   % ... or a4paper or a5paper or ... 
%\geometry{landscape}                % Activate for for rotated page geometry
%\usepackage[parfill]{parskip}    % Activate to begin paragraphs with an empty line rather than an indent
\usepackage{graphicx}
\usepackage{amssymb}
\usepackage{epstopdf}
\DeclareGraphicsRule{.tif}{png}{.png}{`convert #1 `dirname #1`/`basename #1 .tif`.png}

%\usepackage{alphamacros}
%\usepackage{analysismacros}
%\usepackage{formatmacros}
%\usepackage{linalgmacros}
%\usepackage{mymathmacros}
%\usepackage{refmacros}
%
\newcommand{\probbox}[1]{
	\raisebox{-\height}{
		\fbox{
	\begin{minipage}{\textwidth} #1\end{minipage}
			}
		}
	}
\newcommand{\Lap}[1]{\ensuremath{{\mathcal{L}}\left[#1\right]}}

\title{Solutions to Math 51 Final F21}
%\author{The Author}
%\date{}                                           % Activate to display a given date or no date

\begin{document}
\maketitle
%\section{}
%\subsection{}


\probbox{\textbf{II.2}(a) Given the complex eigenvector for $A$ corresponding to the complex eigenvalue
$\lambda=2+i$:
\begin{equation*}
	\vec{v} =
		\begin{bmatrix}
			2-i\\5
		\end{bmatrix}
\end{equation*}
find the general solution of (H)  $DX=AX$
}\\

The complex solution to (H) is
\begin{equation*}
	e^{2t}(\cos t +i \sin t)\begin{bmatrix} 2-i\\5\end{bmatrix}
		=e^{2t}\begin{bmatrix}
			2\cos t+\sin t\\
			5\cos t
		\end{bmatrix}
		+i e^{2t}\begin{bmatrix}
			-\cos t+2\sin t\\
			5\sin t
		\end{bmatrix}
\end{equation*}
so the real and imaginary parts of this generate the general solution
\begin{equation*}
	X(t)=C_{1}e^{2t}\begin{bmatrix}
			2\cos t+\sin t\\
			5\cos t
		\end{bmatrix}
	+C_{2 } e^{2t}\begin{bmatrix}
			-\cos t+2\sin t\\
			5\sin t
		\end{bmatrix}.
\end{equation*}
\\
\probbox{
	(b) Find the solution to (H) satisfying the initial condition
	\begin{equation*}
		X(0)=\begin{bmatrix}1\\1\end{bmatrix}.
	\end{equation*}
}\\

The value at $t=0$ of the general solution given above is
\begin{equation*}
		X(0)=C_{1}e^{0}\begin{bmatrix}
			2\cos 0+\sin 0\\
			5\cos 0
		\end{bmatrix}
	+C_{2 } e^{0}\begin{bmatrix}
			-\cos 0+2\sin 0\\
			5\sin 0
		\end{bmatrix}
		=C_{1}\begin{bmatrix}
			2\\
			5
		\end{bmatrix}
		+C_{2}\begin{bmatrix}
			-1\\
			0
		\end{bmatrix};
\end{equation*}
setting this equal to the desired initial condition yields the system of equations
\begin {align*}
	2C_{1}-C_{2}&=1\\
	5C_{1}+0C_{2}&=1
\end{align*}
which can be solved by either reducing the augmented matrix
\begin{equation*}
	\begin{bmatrix}
		2&-1&|1\\
		5&0&|1
	\end{bmatrix}
\end{equation*}
or Cramer's Rule, or simply by noting that the second equation says
%\begin{align*}
	$C_{1}=\frac{1}{5}$, 
	and substituting into the first equation yields
	$\frac{2}{5}-C_{2}=1$
	or
	$C_{2}=-\frac{3}{5}$..
%\end{align*}
Thus the desired solution of (H) is
\begin{equation*}
	X(t)=
	\frac{1}{5}e^{2t}\begin{bmatrix}2\cos t+\sin t\\5\cos t\end{bmatrix}
	-\frac{3}{4}e^{2t}\begin{bmatrix}-\cos t+2\sin t\\5\sin t\end{bmatrix}
	=e^{2t}\begin{bmatrix}\cos t-\sin t\\\cos t-3\sin t\end{bmatrix}.
\end{equation*}

\noindent I graded this on the basis of 10 for (a) (5 each for the two vector functions) and 5 for (b).


\probbox{\textbf{II.5} (a) Find the inverse Laplace transform
\begin{equation*}
	{\mathcal{L}}^{-1}\left[\frac{3s^{2}+s+1}{(s+1)(s^{2}+2)}\right].
\end{equation*}
}\\

The partial fraction decomposition has the form
\begin{equation*}
	\frac{3s^{2}+s+1}{(s+1)(s^{2}+2)}=\frac{A}{s+1}+\frac{Bs+C}{s^{2}+2};
\end{equation*}
combining over a common denominator and matching coefficients leads to
\begin{equation*}
	\begin{array}{lrrrcr}
		s^{2} \text{ terms}:&A&+B&&=&3\\
		s \text{ terms}:&&B&+C&=&1\\
		\text{constant terms}:&2A&&+C&=&1
	\end{array}
\end{equation*}
We can solve the first (respectively, second) equation for $A$ (respectively, $C$) in terms of $B$:
\begin{align*}
	A&=3-B\\
	C&=1-B
\end{align*}
and substituting into the third equation yields
\begin{align*}
	(6-2B)+(1-B)&=1\\
	-3B&=-6\\
	B&=2\\
	A&=1\\
	C&=-1
\end{align*}
so
\begin{equation*}
	\frac{3s^{2}+s+1}{(s+1)(s^{2}+2)}=\frac{1}{s+1}+\frac{2s-1}{s^{2}+2}.
\end{equation*}
Then the inverse transform is
\begin{align*}
	{\mathcal{L}}^{-1}\left[\frac{3s^{2}+s+1}{(s+1)(s^{2}+2)}\right]&=%\\
	{\mathcal{L}}^{-1}\left[\frac{1}{s+1} \right]+{\mathcal{L}}^{-1}\left[ \frac{2s}{s^{2}+2}\right]-{\mathcal{L}}^{-1}\left[\frac{1}{s^{2}+2} \right]\\
	&=e^{-t}+2\cos t\sqrt{2} -\frac{1}{\sqrt{2}}\sin t\sqrt{2}.
\end{align*}
\\
\newpage
\probbox{
(b) Express the Laplace Transform of the solution of the o.d.e.
%\begin{equation*}
	$(D^{2}+D+1)x=1$
satisfying the initial conditions $x(0)=0$ and $x'(0)=1$
as a function of $s$:
}\\

By the first differentiation formula, applying the Laplace Transform
to both sides of the problem yields
\begin{align*}
	\Lap{D^{2}x}+\Lap{Dx}+\Lap{x}&=\Lap{1}\\
	\lbrace s^{2}\Lap{x}-s x(0)-x'(0)\rbrace
		+\lbrace s\Lap{x}-x(0)\rbrace+\Lap{x}
		&=\Lap{1}\\	
	s^{2}\Lap{x}-1
		+s\Lap{x}+\Lap{x}
		&=\frac{1}{s}\\
	(s^{2}+s+1)\Lap{x}&=1+\frac{1}{s}=\frac{1+s}{s}\\
	\Lap{x}&=\frac{1+s}{s(s^{2}+2+1)}	
\end{align*}
\\
I graded this as 10 for (a) (5 for partial fractions, 5 for inverse transform) and 5 for (b).











\end{document}