\documentclass{article} 

\usepackage{amssymb,amsmath}
\usepackage[margin=2cm]{geometry}

\newcommand\ms {\medskip}
\newcommand\bs {\bigskip}
\newcommand\dsp{\displaystyle}
\newcommand\cc{{\mathcal{C}}}
\newcommand\rr{{\mathbb{R}}}
\newcommand\rone{{\mathbb{R}}}
\newcommand\rtwo{{\mathbb{ R}^2}}
\newcommand\rn{{\mathbb{R}^n}}
\newcommand\rrm{{\mathbb{R}^m}}
\newcommand\V{\mathcal{V}}

\newcommand\nn{{\mathbb{N}}}
\renewcommand\ss{{\mathbb{S}}}

%\newcommand{\cl}{\operatorname{cl}}
\newcommand{\bd}{\operatorname{bd}}
\newcommand\zz{{\mathbb{Z}}}
\newcommand\ii{{\mathbb{I}}}
\newcommand\qq{{\mathbb{Q}}}
\newcommand\eps{{\epsilon}}
\newcommand\rar{\rightarrow}
\newcommand\st{\hspace{0.4mm}\big|\hspace{0.4mm}}
\newcommand\x{\mathbf{x}}
\newcommand\y{\mathbf{y}}
\newcommand\zero{\mathbf{0}}
\newcommand\xoyo{(x_0,y_0)}

\def\dsp{\displaystyle}
\def\rr{{\mathbb R}}
\def\R{{\mathbb R}}
\def\rone{{\mathbb R}}
\def\rtwo{{\mathbb R^2}}
\def\rn{{\mathbb R^n}}
\def\rm{{\mathbb R^m}}
\def\nn{{\mathbb N}}
\def\zz{{\mathbb Z}}
\def\Q{{\mathbb Q}}
\def\rar{\rightarrow}
\def\st{\,\big|\,}
\def\eps{\epsilon}
\def\acc{\operatorname{acc}}
\def\sup{\operatorname{sup}}
\def\inf{\operatorname{inf}}
\def\intt{\operatorname{int}}
\def\bd{\operatorname{bd}}
\def\cl{\operatorname{cl}}
\def\intab{\int_a^b}
\def\gap{\operatorname{gap}}
\def\vx{\vec {\mathbf{x}}}
\def\vy{\vec {\mathbf{y}}}
\def\vL{\vec {\mathbf{L}}}
\def\vM{\vec {\mathbf{M}}}
\def\ms{\medskip}
\def\ss{\smallskip}
\def\bfu{\mathbf u}
\def\bfv{\mathbf v}

\setlength\parindent{25pt}



\begin{document}


\noindent
PRINT your name:  \framebox(230,30){}\\

\smallskip
\noindent
Practicum Instructor: \framebox(100,30){} (Choose from Hasselblatt, McNinch, Smith, Tu)


\begin{center}
Math 51 \hfill Differential Equations \hfill February 14, 2022\\
\hfill Exam 1 (100 points) \hfill noon--1:20 p.m.
\end{center}

\medskip

This is a closed-book exam.  No books, notes, or calculators are
permitted.  At the end of the exam, scan your exam as a single pdf
file and upload them to Gradescope.

We will scan your exam submission and upload it to \emph{Gradescope} for
marking -- \emph{you do not need to scan your exam}. You should write
your name at the top of each page, as indicated (\emph{especially} if you
remove the staples from your exam booklet).

For the partial credit problems, always \emph{show your work}. Try to
fit this work in the \emph{available space} if possible.  There is a
blank page at the back of your exam for use as scratch paper. If you
need more space for a solution, please write clearly \emph{in the
  indicated space} that your solution continues later.

At the end of the exam you are required to sign your exam. With your
signature, you pledge that you have neither given nor received
assistance on this exam.

Students found violating the pledge will be reported to
the Dean of Student Affairs.

You must show your work in all the questions that require
calculations, explanations, or proofs.  Simplify all answers.


%\setcounter{page}{1}


\newpage

\begin{enumerate}
 \setlength{\itemsep}{5mm}

 % 1.  

\item (20 points) Short-answer questions.  No work needs to be shown, as only the answer will be graded.

  \begin{enumerate}
    \setlength{\itemsep}{5mm}


    \item Consider the differential equation
      \[
        (t-3)^2 x'' + \frac{1}{t-2} x' + 2x = 0.
        \]
        Find the largest open interval containing $t=0$ on which the
        o.d.e.\ is normal.
        
        \hfill \framebox(250,30){}

        \item  True or False.  Let $h_1, h_2, h_3$ be solutions of a normal
          third-order linear differential equation on an open
      interval $I$.  Then the Wronskian $W[h_1, h_2, h_3](t_0) = 0$
      at one point $t_0 \in I$ if and only if $W[h_1, h_2, h_3](t)
      =0$ at every point $t \in I$.

      \hfill \framebox(250,30){}

      \item True or False.  Let $h_1, h_2, h_3$ be arbitrary infinitely differentiable
        functions on an open interval $I$.  Then the Wronskian $W[h_1, h_2, h_3](t_0) = 0$
      at one point $t_0 \in I$ if and only if $h_1, h_2, h_3$
      are linearly dependent on $I$.

      \hfill \framebox(250,30){}

    \item The order of the o.d.e. $t^3 (x'')^3 + 3x' +tx^4 = 0$ is

      \hfill \framebox(250,30){}

        \item True or False.  The linear o.d.e $t^3 x'' + 3x' + x + t
          = 0$ is homogeneous.

          \hfill \framebox(250,30){}
\end{enumerate}
    
\newpage
  
\item (15 points) Find the general solution of
  \[
    t^2 \frac{dx}{dt} + 2tx = e^{2t}, \quad t >0.
  \]

  \item (10 points) Determine whether each collection of functions is linearly
    independent or dependent on the interval $-\infty < t < \infty$.
    Justify your answer.
    \begin{enumerate}
    \item $f_1(t) = e^t, \quad f_2(t) = t e^t, \quad f_3(t) = 1$.
    \item $g_1(t) = t^2, \quad g_2(t) = - t^2$.
      \end{enumerate}

\item (20 points)  Write the general solution for each differential
  equation below.
  \begin{enumerate}
  \item $D(D^2 -9) (D^2 -2D-1) x = 0$.
  \item $(D^2 -2D + 10)^2 x = 0$.
  \end{enumerate}

\item (10 points) Find a minimal annihilator for the function
    $5 e^{3t} + 4 e^{-2t} + 3 e^{-5}\cos 3t$ or state why there isn't
    one.
  \item (10 points) Evaluate $(D+5)^3 (t^3 - 4t^2 + 2t -4) e^{-5t}$.
  
  \item (15 points) Consider the equation $t^2 x'' - t x' + x =0$ on
    the interval $(0, \infty)$.
    \begin{enumerate}
      \item Show that $h_1(t) = t$ and $h_2(t) = t \ln t$ are
        solutions on this interval.
        \item Show that these two solutions do or do not generate the
          general solution.
        \end{enumerate}

      \end{enumerate}

      \end{document}
 





  
% \item[(c)]  ( pts)  Calculate
%   \[
%     \det \begin{bmatrix}
%       4 & 0 &2 \\
%       1 & 1 & 0 \\
%       2 & -1 & 1
%     \end{bmatrix}.
%     \]
 
          
  0.4pt depth 0pt width 1 true in .
      \end{enumerate}

      \item[(d)] (3 pts) Suppose
          \begin{align*}
            u_1 + 2 u_2 + 3 u_3 &= a,\\
            4u_1 + 5 u_2 + 6 u_3 &= b,\\
            u_1 - \phantom{5} u_2 + \phantom{6} u_3 &= c,
          \end{align*}
          where it is given that the determinant of the coefficient matrix is nonzero.
          Write down the formula for $u_3$ in terms of determinants.
          Do not evaluate the determinants.
  
         \vfill\eject

          \item[(e)] (4 pts) Determine the largest interval containing 2 on
            which the differential equation below is normal:
            \[
              (t^2 -1) \frac{d^2 x}{dt^2} - 3 \frac{dx}{dt} + x = t.
            \]

            \vspace*{2in}

             \item[(f)] (4 pts)  The atoms of a radioactive substance tend to decompose into
      atoms of a more stable substance at a rate proportional to the
      number $x = x(t)$ of unstable substance present.  Set up a
      differential equation for $x = x(t)$; in your differential equation, indicate whether the constant is positive or negative.

      \vspace*{2in}

    \item[(g)] (5 pts) Find a rational root of the polynomial $P(D) = 2 D^3 -5D^2 + 1$.

      \vspace*{1.5in}

        \end{enumerate}

        \newpage

        
       

    \item (6 points)  Find the general solution of
      \[
        ( D^2 + 2D + 2 ) x = 0.
      \]

      \newpage

  \item (7 points)  Make a \textit{simplified} guess for a particular solution of
    the differential equation
    \[
      (D+2)^7 (D^2+1)^6 x = t e^{-2t} + \cos t.
    \]
    Do not solve for the coefficients.

    \newpage

   
    \item (15 points) Solve the initial-value problem
      \[
        t x' - x = t^3, \quad x(1) = 0,
      \]
      on the interval $(0,\infty)$.

      \newpage

      \item (10 points) Using the definition of linear independence, show that
        $\cos t, \sin t, \sin 2t$ are linearly independent.

        \newpage

        \item (10 points) Use the exponential shift formula to compute the
          following expressions:
          \begin{enumerate}
          \item[(a)] (5 pts) $(D^2 -6D - 1)[e^{3t} \cos t]$

            \vspace*{2.8in}
            
          \item[(b)] (5 pts) $(D-3)^4 [ t^4 e^{3t} ]$

            \vspace*{2.8in}
          \end{enumerate}


\end{enumerate}



\vfill
\noindent
\tred{STOP!  Have you submitted the Quiz portion of the exam on Canvas?}

\bigskip
\bigskip

\noindent
PLEDGE:  I pledge that during this exam I have neither given nor received assistance or cheated in any other way.

\bigskip
\noindent
Signature: \framebox(250,30){}

\begin{center}
(End of Exam)
\end{center}

\end{document}
