\documentclass[12pt]{article}
%\pagestyle{empty}
% \usepackage[left = 0.9in, right = 0.9in, top = 0.9in, bottom =
% 0.9in]{geometry}
\usepackage[margin=1.2in]{geometry}
\usepackage{amsmath,amssymb,comment, enumerate,hyperref,xcolor,bm,graphicx}

%\usepackage{soul}
 \usepackage{palatino}
 \usepackage{mathpazo}
 
\def\A{\mathbf{A}}
\def\B{\mathbf{B}}
\def\C{\mathbf{C}}
\def\P{\mathbf{P}}
\def\D{\mathbf{D}}
\def\T{T}
\def\V{\mathbf{V}}
\def\U{\mathbf{U}}
\def\X{\mathbf{X}}
\def\Z{\mathbf{Z}}
\def\c{\mathbf{c}}
\def\e{\mathbf{e}}
\def\x{\mathbf{x}}
\def\z{\mathbf{z}}
\def\v{\mathbf{v}}
\def\y{\mathbf{y}}
\newcommand{\cL}{{\mathcal L}}

\usepackage{color}
\newcommand{\tred}[1]{{\color{red}{#1}}}


\begin{document}
\noindent
{\large {\bf Math 51}   \hfill  {\bf Homework 12} \hfill   Spring 2022}   

\bigskip
\bigskip

\centerline{\textbf{Readings for the Week of April 11, 2022}}

\medskip
\noindent
Martin Guterman and Zbigniew Nitecki,
  \textit{Differential Equations: A First Course}, 3rd edition.  ISBN: 81-89617-20-6.
\begin{enumerate}
    \item[\S 5.2] The Laplace Transform: Definitions and Basic Calculations
    \item[\S 5.3] The Laplace Transform and Initial-Value Problems
    \end{enumerate}


 \bigskip
\noindent
\centerline{\textbf{Problem Set 12}}
\centerline{(Due \tred{Monday, April 25}, 2022, at 11:59 p.m.)}

\medskip

\noindent
For a 10\% penalty on your grade, you may hand in the problem set
late, until Tuesday, April 26, 2022, 11:59 p.m.

\medskip

\begin{enumerate}[1.]
\setlength{\itemsep}{5mm}

% Similar to \S 5.2, p. 421, #4.
\item \textbf{(Laplace transform from the definition)} \\
Let $f(t) = te^{2t}$.  Calculate the Laplace transform
  $F(s) = \cL[f(t)]$ directly
  from the definition and indicate the values of $s$ for which the
  integral defining $F(s)$ converges.

  % Similar to \S 5.2, p. 421, #12, #14, #15.
\item \textbf{(Laplace transform)} \\
For each of the following functions, calculate its Laplace
  transform $F(s) = \cL[f(t)]$ using the linearity of $\cL$ together
  with the basic formulas summarized at the end of \S 5.2.
  
  \bigskip
  \begin{enumerate}
    \setlength{\itemsep}{2mm}
  \item[(a)] $f(t) = 2t + e^{-4t} - 3\cos 5t$.
  \item[(b)] $f(t) = e^{3t+2}$.
    \item[(c)] $f(t) = (t+2)(t+3)$.
      \end{enumerate}

       % \S 5.2, p. 422, #19 (similar), #24.
  \item \textbf{(Inverse transform)} \\
  For each of the following functions, calculate its inverse
  transform $f(t) = \cL^{-1}[F(s)]$ using the linearity of $\cL^{-1}$ together
  with the basic formulas summarized at the end of \S 5.2.
  \begin{enumerate}
    \setlength{\itemsep}{2mm}
  \item[(a)] $F(s) =\displaystyle \frac{1}{3s+1}$.
 
  \item[(b)] $F(s) = \displaystyle \frac{2}{s^2+4} -\frac{10}{s^4}
    +\frac{1}{s}$.
    
    \end{enumerate}

    % \S 5.3, pp. 432--433, #2, #4, #6.
    \item \textbf{(First differentiation formula)} \\
    Use the first differentiation formula to find an expression
      for the Laplace transform $\cL[x]$, where $x$ is the solution of
      the given initial-value problem.
    
      \begin{enumerate}
        \setlength{\itemsep}{2mm}
      \item $(D - 1) x = e^{2t}, \quad x(0) =2$.
      \item $(D^2 -1) x = e^{2t}, \quad x(0) = 0, \ \ x'(0) = 1$.
      \item $(D^2 + 1) x = \cos 3t, \quad x(0)=0, \ \ x'(0) = 3$.
      \end{enumerate}

      % \S 5.3, pp.~433, \#12.
     \item \textbf{(Partial fraction decompostion)} \\
      Find the inverse transform of $F(s) = \displaystyle \frac{s+4}{s^2
         + 4s +3}$.

       % \S 5.3, pp.~433, \#20.
      \item \textbf{(Initial-value problem)}\\
       Use the Laplace tranform to solve the initial-value problem:
       \[
         (D^2 + 4) x = t, \quad x(0)= -1,\ \  x'(0) = 0.
         \]
     


\end{enumerate}

\bigskip

\centering{(End of Homework 11)}

\end{document}