\documentclass[12pt]{article} 
\usepackage{amsmath,amssymb,amscd} 
%\usepackage{pstricks}
%\usepackage[pdftex]{graphicx}
%\thispagestyle{empty}

\usepackage[margin=.9in]{geometry}

%\addtolength{\textwidth}{1in}
%\addtolength{\oddsidemargin}{-.5in}
%\addtolength{\evensidemargin}{-.5in}
%\addtolength{\textheight}{1in}
%\addtolength{\topmargin}{-.5in}

% \usepackage{mathptmx}
% \renewcommand{\rmdefault}{ptm}		% set Times as the default text font
 \usepackage{palatino}
 \usepackage{mathpazo}

 %\input manifoldsmacros
 
%\thispagestyle{empty}

\usepackage{color}
\newcommand{\tred}[1]{{\color{red}{#1}}}


\begin{document}

%\setcounter{page}{4}

\noindent
Math 51~~~~~~~~~~~~~~~~~~~ \hfill Differential Equations \hfill February 3--4, 2022 
 \centerline{Worksheet 2}

\begin{enumerate}
\item[1.] The following example shows that in some physical situations,
  nonuniqueness is natural and obvious, not pathological.\footnote{See
  Steven H. Strogatz: Nonlinear Dynamics and Chaos: With Applications to Physics, Biology, Chemistry, and Engineering}

  Consider a water bucket with a hole in the bottom. If you see an empty bucket with a puddle beneath it, can you figure out when the bucket was full? No, of course not! It could have finished emptying a minute ago, ten minutes ago, or whatever. The solution to the corresponding differential equation must be nonunique when integrated backwards in time.

Here's a crude model of the situation. Let \(h(t)=\)height of the water
remaining in the bucket at time \(t\); \(a =\)area of the hole; \(A
=\)cross-sectional area of the bucket (assumed constant); \(v(t)
=\)velocity of the water passing through the hole.
\begin{enumerate}
\item Show that \(av(t) = Ah(t)\). What physical law are you invoking?
\item To derive an additional equation, use conservation of energy. First, find the
change in potential energy in the system, assuming that the height of the
water in the bucket decreases by an amount \(\Delta h\) and that the water has density \(\rho\) . Then find the kinetic energy transported out of the bucket by the escaping water. Finally, assuming all the potential energy is converted into kinetic energy, derive the equation \(v^2 = 2gh\).
\item Combining the previous two items, show that \(\dot h=-C\sqrt h\),  where \(C=\sqrt{2g}\dfrac aA\) and \(\dot h=\frac{d}{dt}h\).
\item Given \(h(O) =0\) (bucket empty at \(t =0\) ), show that the solution for \(h(t)\) is nonunique in backwards time, i.e., for \(t < 0\).
\end{enumerate}
\newpage
\item[2.] 
Consider the ODE   
\[
x''+x'-6x=3t+2.
\]
\begin{enumerate}
\item Find all linear solutions.
\item Find all linear solutions of the associated homogeneous equation.
\item Find all solutions of the associated homogeneous equation that are of
  the form \(\sin at\) or \(\cos bt\).
\item Find all solutions of the associated homogeneous equation that are of
  the form \(e^{ct}\).
\item Use the Wronskian test to check whether the solutions from the
  previous three items generate the general solution.
\item Find the general solution of  \(x''+x'-6x=3t+2\).
\end{enumerate}
\end{enumerate}\end{document}





\end{document}
