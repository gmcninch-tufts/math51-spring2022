\documentclass[12pt]{article} 
\usepackage{amsmath,amssymb,amscd,enumerate,mathtools} 
%\usepackage{pstricks}
%\usepackage[pdftex]{graphicx}
%\thispagestyle{empty}

\usepackage[margin=.9in]{geometry}

%\addtolength{\textwidth}{1in}
%\addtolength{\oddsidemargin}{-.5in}
%\addtolength{\evensidemargin}{-.5in}
%\addtolength{\textheight}{1in}
%\addtolength{\topmargin}{-.5in}

% \usepackage{mathptmx}
% \renewcommand{\rmdefault}{ptm}		% set Times as the default text font
 \usepackage{palatino}
 \usepackage{mathpazo}

 %\input manifoldsmacros
 
%\thispagestyle{empty}

\usepackage{color}
\newcommand{\tred}[1]{{\color{red}{#1}}}


\begin{document}

%\setcounter{page}{4}

\noindent
Math 51~~~~~~~~~~~~~~~~~~~ \hfill Differential Equations \hfill March 3--4, 2022 
 \centerline{Worksheet Week 7}

% \begin{enumerate}[1. ]
% \item The vector-valued function \(\displaystyle\vec
%   x=\begin{bmatrix}e^{-3t}\\e^{3t}\end{bmatrix}\) determines a parametrized
%   curve. Sketch the curve and indicate with an arrow the direction in which
%   the point \(\vec x(t)\) moves along the curve as \(t\) increases.
% \item Find the solution of the initial-value problem
%   \[D\vec x=\begin{bmatrix}1&0\\0&2\end{bmatrix}\vec x\qquad\vec x(0)=\begin{bmatrix}1\\1\end{bmatrix}
%   \]
%   The solution determines a parametrized
%   curve. Sketch the curve and indicate with an arrow the direction in which
%   the point \(\vec x(t)\) moves along the curve as \(t\) increases.
% \newpage
\begin{enumerate}[1.]
\item Consider the system of ordinary differential equations
\[
  \begin{bmatrix}x_1'\\x_2'\\x_3'\end{bmatrix}
    =
  \begin{bmatrix*}[r]5&-3&0\\3&-5&0\\0&1&2\end{bmatrix*}
  \begin{bmatrix}x_1\\x_2\\x_3\end{bmatrix}
      +
      \begin{bmatrix}0\\0\\4\end{bmatrix}.
    \]

Consider the solutions:    
\(\left \{.\begin{aligned} 
x_1&=&(6c_1+6c_3)&e^{4t}+&(-2c_2+2c_3)&e^{-4t}\\
x_2&=&(2c_1+2c_3)&e^{4t}+&(-6c_2+6c_3)&e^{-4t}\\
x_3&=&(c_1+c_3)  &e^{4t}+&(c_2-c_3)   &e^{-4t}-2\\
\end{aligned} \right.\)
    
\begin{itemize}
\item Describe these solutions in the form \(\mathbf{p} + c_1\mathbf{h}_1 + c_2 \mathbf{h}_2 + c_3 \mathbf{h}_3\)
  \vfill
\item Check directly that \(\mathbf{p}\)   and \(\mathbf{h}_1\) are actually solutions.
  \vfill
\item Decide whether this collection of solutions is \emph{complete.}
  \vfill
\end{itemize}

\newpage
\item For each of the following scenarios for vectors \emph{in the plane} either draw a picture of such a scenario or explain why this can't be done.
\begin{enumerate}
\item One linearly independent vector
\vfill    
\item One linearly dependent vector
\vfill    
\item Two linearly independent vectors
\vfill    
\item Two linearly dependent vectors
\vfill    
\item Three linearly independent vectors
\vfill    
\item Three linearly dependent vectors
\vfill  
\end{enumerate}
\newpage
\item Consider vectors \(\mathbf{v}_1,\mathbf{v}_2,\mathbf{v}_3,\mathbf{v}_4\).

  \begin{itemize}
  \item   Show that if \(\mathbf{v}_1 = \mathbf{v}_4\), then \(\mathbf{v}_1,\mathbf{v}_2,\mathbf{v}_3,\mathbf{v}_4\) is linearly dependent.
    \vfill
  \item Show that if \(\mathbf{v}_3 = \mathbf{0}\)\),  then \(\mathbf{v}_1,\mathbf{v}_2,\mathbf{v}_3,\mathbf{v}_4\) is linearly dependent.
    \vfill
  \end{itemize}
\eject\end{enumerate} 





\end{document}
