\documentclass[10pt]{article}
%\pagestyle{empty}
\usepackage[left = 0.9in, right = 0.9in, top = 0.9in, bottom = 0.9in]{geometry}
\usepackage{amsmath,amssymb,comment, enumerate,hyperref,xcolor,bm,graphicx}
\def\A{\mathbf{A}}
\def\B{\mathbf{B}}
\def\C{\mathbf{C}}
\def\P{\mathbf{P}}
\def\D{\mathbf{D}}
\def\T{T}
\def\V{\mathbf{V}}
\def\U{\mathbf{U}}
\def\X{\mathbf{X}}
\def\Z{\mathbf{Z}}
\def\c{\mathbf{c}}
\def\e{\mathbf{e}}
\def\x{\mathbf{x}}
\def\z{\mathbf{z}}
\def\v{\mathbf{v}}
\def\y{\mathbf{y}}
\def\w{\mathbf{w}}
\def\h{\mathbf{h}}

\begin{document}
\noindent
{\large {\bf Math 51}   \hspace{13em}  {\bf Worksheet 8} \hspace{11em}   \hspace{2em}   


\begin{enumerate}
\item 
\begin{enumerate}[(a)]
\item Find the eigenvalues of the following matrices:
\begin{equation*}
A = \begin{bmatrix}
2 & 0 & 1\\
1 & 2 & 1\\
-1 & 0 & 1
\end{bmatrix},~~
B = \begin{bmatrix}
2&1&-1\\
0&2&0\\
1&1&1
\end{bmatrix}
\end{equation*}
\vfil
\item Describe how to obtain matrix \(B\) from matrix \(A\). Are their
  their eigenvalues related? If so, how?
% \item Find the eigenvalues of the following matrix:
% \begin{equation*}
% C = \begin{bmatrix}
% 1&0&0&0\\
% 2&-1&0&0\\
% 3&3&0&0\\
% 4&0&4&-2
% \end{bmatrix}
% \end{equation*}
\end{enumerate}

\vfil

\item Find four linearly independent  eigenvectors for the following matrix:
\begin{equation*}
A = \begin{bmatrix}
1&0&0&0\\
0&2&0&0\\
0&0&3&0\\
0&0&0&4
\end{bmatrix}
\end{equation*}

\vfil
\newpage

\item The eigenvalues of the following matrix $A = \begin{bmatrix}
    1&1&1\\0&2&0\\0&0&1\end{bmatrix}$ are 1 and 2. The vector
   $\v = \begin{bmatrix} 1\\0\\0
\end{bmatrix}$ is an eigenvector for \(\lambda = 1\) and $\w = \begin{bmatrix}
1\\1\\0
\end{bmatrix}$ is an eigenvector for \(\lambda = 2\).
\begin{enumerate}
\item The eigenvalue \(\lambda = 1\) has multiplicity 2. However, show that in this case,
  there are \emph{not} two linearly independent eigenvectors for \(\lambda = 1\).
  \vfil
\item Let $\h_1(t) = e^t\v$ and $\h_2(t)=e^{2t}\w$. Explain why, for
  any $c_1,c_2$, the vector-valued function $\mathbf{x}(t) =
  c_1\h_1(t)+c_2\h_2(t)$ is a solution to \(D \mathbf{x} = A
  \mathbf{x}\), but is not the \emph{general} solution.  \vfil
\item Let $\x(t)$ be a solution to $D\x=A\x$ with
  $\x(0)=\begin{bmatrix} 1\\-1\\0
         \end{bmatrix}$. Find scalars $c_1,c_2$ such that $\x(t) = c_1\h_1(t)+c_2\h_2(t)$.

  \vfil         
\item Let $\x(t)$ be a solution to $D\x=A\x$ with $\x(0)=\begin{bmatrix}
0\\2\\1
\end{bmatrix}$. Show that there are no scalars $c_1,c_2$ for which $\x(t) = c_1\h_1(t)+c_2\h_2(t)$.
\end{enumerate}

\vfil


\newpage

\item Solve the initial value problem 
\begin{equation*}
D\x = \begin{bmatrix}
2&-1\\
3&1
\end{bmatrix}\x;~~\x(0)=\begin{bmatrix}
1\\
1
\end{bmatrix}
\end{equation*}

\vfil
\end{enumerate}

\end{document}